%------------------------------------------------------------------------------
% Template file for the submission of papers to IUCr journals in LaTeX2e
% using the iucr document class
% Copyright 1999-2003 International Union of Crystallography
% Version 1.2 (11 December 2002)
%------------------------------------------------------------------------------
%
\documentclass[preprint, pdf]{iucr}              % DO NOT DELETE THIS LINE
                   \def\href#1{\relax}\let\foo\caption
\ifPDF
  \RequirePackage{hyperref}
  \PassOptionsToPackage{pdftex,bookmarksopen,bookmarksnumbered}{hyperref}
  \voffset=-0.5in
\fi
\let\caption\foo

\usepackage{graphicx}
\usepackage[T1]{fontenc}
\usepackage[utf8]{inputenc}
 \usepackage{amsmath}
%\usepackage{eps2pdf}

 \paperprodcode{a000000}      % Replace with production code if known
 \paperref{xx9999}            % Replace xx9999 with reference code if known
 \papertype{IU}               % Indicate type of article
 \paperlang{english}          % Can be english, french, german or russian
 \journalcode{S}             % Indicate the journal to which submitted
 \journalyr{2017}
 \journalreceived{\relax}
 \journalaccepted{\relax}
 \journalonline{\relax}

\begin{document}                  % DO NOT DELETE THIS LINE

\title{Speeding-up diffraction tomography by accelerating image decompression}
\shorttitle{GPU decompression of CBF images}

 \cauthor[a]{J.}{Kieffer}{jerome.kieffer@esrf.eu}
 
 \aff[a]{ESRF, The European Synchrotron, CS40220, 38043 \city{Grenoble}
 Cedex 9, \country{France}}
 \shortauthor{Kieffer}


\keyword{Powder diffraction}
\keyword{Diffraction imaging}
\keyword{Pilatus detector}
\keyword{pyFAI}
\keyword{FabIO}
\keyword{Silx}



\maketitle                        % DO NOT DELETE THIS LINE

\begin{synopsis}
\end{synopsis}

\begin{abstract}

Diffraction imaging is an X-Ray imaging method which uses the cristallinity
as signal. 
A pencil beam is raster-scanned onto a sample and the (powder) diffraction
signal is recoreded by a large area detector.
In this experiment, especially with current pixel-detector, the speed of
measurement is limited by the storage capacity and online data reduction
hardly ever done. 

This contribution presents the benchmarking of a typical data
analysis pipeline for a diffraction imaging like the one performed at ESRF ID15
and proposes some disruptive techniques to decode CBF images using the
computational power of GPUs. 

\end{abstract}


\section{Introduction}

Diffraction imaging is an X-Ray imaging method which uses the cristallinity
as signal where a pencil beam is raster-scanned onto a sample and the
(powder) diffraction signal is recoreded by a large area detector.
It turns out the current detector are already
fast enought to fill the temporary storage and data analysis \emph{is} the
limiting factor for the whole experiment. 

The analysis of this experiment is often a simple azimuthal regrouping of the
diffraction image into a powder pattern with the intensity given as function of
the scattering vector $q=4 \pi sin (2\theta/2)/\lambda$. 
More complex analysis are possible and often desirable but they are even more
resource intensive.

We will focus in the second section on the setup of the Materials beamline
(ID15a) of the ESRF and perform a complete benchmarking of the data analysis 
pipeline used. This will to highlight various bottle-necks of the data-analysis
chain.
To address them, a new CBF image decompression algorithm has been developped and
is presented in section 3. 

\section{Diffraction imaging data analysis pipeline}

\subsection{Beamline hardware}

\subsubsection{Pilatus3 2M CdTe detector}

The ID15a beamline at ESRF uses mainly a Pilatus3 2M detector with a 1000 $\mu
m$ CdTe sensor, manufactured by Dectris \cite{dectris}. 
The detector is made-up of 8\times3 Pilatus modules (100 kpixel each).
Unlike Silicon-based sensors, there are two Cd-Te waffers bump-bound to every
Pilatus-module, with a gap of 3 pixels between the wafers.

This detector is sold with a detector-PC which is in charge of saving the
the images. 
This detector-PC comes with a 10Gbit/s network card and is directly connected to
the data analysis server.

The detector is advertized to operate at 250 frames per seconds. 
Each frame being composed of 24 pilatus modules, this makes 2.4 megapixel
images, which are 32-bits integers (the dynamic range is only 20 bits).
If one saves the images as uncompressed raw stream this makes 19.2 gigabits per
second to transfer.
Compression is hence mandatory to transfer the acquired data through the 10
gigabit network interface.
Dectris uses the CIF binary format (cbf) for their Pilatus detectors with
byte-offset compression which provides a compression factor close to 4\times. 
An alternative compression scheme used in novel Eiger detector is the LZ4 but
the compression factor is much less, arround 2\times (and variable) which makes
this option un-applicable for operating the Pilatus detector continuously over
an extended period of time.

\subsubsection{Data analysis computer}

The detector computer is directly attached to the netwok interface of the data
analysis computer. 
This data analysis computer has two processors Intel Xeon E5-2643 v3, 6
cores/socket 20Mbyte of cache, and 128 gigabytes of memory. 
There is additionnaly two 10-gigabit network cards, a fast SSD, and a Nvidia
Titan-X graphics, all connected on the PCI-express bus.

\section{Benchmarking of the data-analysis pipeline}

\subsection{Processing of diffraction imaging experiment}


import h5py, fabio, pyFAI

def process(list_of_files, result_file, geometry_file, number_of_bins):
    ai = pyFAI.load(geometry)
	with h5py.File(result_file) as result:
		dataset = result.create_dataset((len(list_of_files), number_of_bins,dtype='float32')
		for index, one_file in enumerate(list_of_files):
		    with fabio.open(one_file) as fimg:
		        data = fimg.data
		    res = ai.integrate1d(data, number_of_bins)
		    dataset[index] = res.intensity
	    result['q'] = res.radial


\section{Decompression of CBF images}

\subsection{Decompression on the processor}

The core idea of the byte-offset compression is to encode only the difference
value with the previous pixel and hope this value is small engouh to fit in a
8-bit (signed) integer. 
Larger values are coded with a special value (-128) which indicates an exception
and subsequent 2 bytes are used to store a 16-bit integrer. 
If the value does not fit in a 16-bit integer a 32-bit exception is signaled and
the value is coded on 32 bits. 
With this encoding, one value is coded with variable size on 1, 3 or 7 bytes,
which makes it very difficult to decompress in a parallel fashion. 

\subsection{Pool of worker startegy} 

\subsection{Parallel decompression on GPU}

\section{Outlook}


\section{Conclusion}

 
\ack{Acknowledgements}
This work was carried out on the request from the Materials diffraction beamline 
at ESRF (ID15a), I would like to thank Gavin Vaughan and Marco Di Michiel, the
scientists of the beamline for providing some test data.
In the instrumentation services and development division (ISDD) from ESRF  we
would like to thank V. Armando Solé, head of data analysis unit and leader of 
the \textit{silx} project, and all our colleagues from the \textit{silx}
project:
Thomas Vincent, Henri Payno, Damien Naudet, Pierre Knobel and Valentin Valls for
their support and ideas.

\bibliographystyle{iucr}
\bibliography{biblio}


\end{document}
