%------------------------------------------------------------------------------
% Template file for the submission of papers to IUCr journals in LaTeX2e
% using the iucr document class
% Copyright 1999-2003 International Union of Crystallography
% Version 1.2 (11 December 2002)
%------------------------------------------------------------------------------
%
\documentclass[preprint, pdf]{iucr}              % DO NOT DELETE THIS LINE
                   \def\href#1{\relax}\let\foo\caption
\ifPDF
  \RequirePackage{hyperref}
  \PassOptionsToPackage{pdftex,bookmarksopen,bookmarksnumbered}{hyperref}
  \voffset=-0.5in
\fi
\let\caption\foo

\usepackage{graphicx}
\usepackage[T1]{fontenc}
\usepackage[utf8]{inputenc}
 \usepackage{amsmath}
%\usepackage{eps2pdf}

 \paperprodcode{a000000}      % Replace with production code if known
 \paperref{xx9999}            % Replace xx9999 with reference code if known
 \papertype{IU}               % Indicate type of article
 \paperlang{english}          % Can be english, french, german or russian
 \journalcode{S}             % Indicate the journal to which submitted
 \journalyr{2017}
 \journalreceived{\relax}
 \journalaccepted{\relax}
 \journalonline{\relax}

\begin{document}                  % DO NOT DELETE THIS LINE

\title{Speeding-up diffraction tomography by accelerating image decompression}
\shorttitle{GPU decompression of CBF images}

 \cauthor[a]{J.}{Kieffer}{jerome.kieffer@esrf.eu}
 
 \aff[a]{ESRF, The European Synchrotron, CS40220, 38043 \city{Grenoble}
 Cedex 9, \country{France}}
 \shortauthor{Kieffer}


\keyword{Powder diffraction}
\keyword{Diffraction imaging}
\keyword{Pilatus detector}
\keyword{pyFAI}
\keyword{FabIO}
\keyword{Silx}



\maketitle                        % DO NOT DELETE THIS LINE

\begin{synopsis}
\end{synopsis}

\begin{abstract}

Diffraction imaging is an X-Ray imaging method which uses the cristallinity
as signal where a pencil beam is raster-scanned onto a sample and the
(powder) diffraction signal is recoreded by a large area detector.
It turns out novel detector are already
fast enought to fill the temporary storage and data analysis \emph{is} the
limiting factor for this kind of experiment. 

In this document the author presents a benchmarking of the data analysis
pipeline of the pipeline used at ESRF ID15a and proposes a new way of
de-compressing CBF images using 

\end{abstract}


\section{Introduction}

Diffraction imaging is an X-Ray imaging method which uses the cristallinity
as signal where a pencil beam is raster-scanned onto a sample and the
(powder) diffraction signal is recoreded by a large area detector.
It turns out the current detector are already
fast enought to fill the temporary storage and data analysis \emph{is} the
limiting factor for the whole experiment. 

The analysis of this experiment is often a simple azimuthal regrouping of the
diffraction image into a powder pattern with the intensity given as function of
the scattering vector $q=4 \pi sin (2\theta/2)/\lambda$. 
More complex analysis are possible and often desirable but they are even more
resource intensive.

We will focus in the second section on the setup of the Materials beamline
(ID15a) of the ESRF and perform a complete benchmarking of the data analysis 
pipeline used. This will to highlight various bottle-necks of the data-analysis
chain.
To address them, a new CBF image decompression algorithm has been developped and
is presented in section 3. 

\section{Diffraction imaging data analysis pipeline}

\subsection{Beamline hardware}

\subsubsection{Pilatus3 2M CdTe detector}

The ID15a beamline at ESRF uses mainly a Pilatus3 2M detector with a 1000 $\mu
m$ CdTe sensor, manufactured by Dectris \cite{dectris}. 
The detector is made-up of 8\times3 Pilatus modules (100 kpixel each).
Unlike Silicon-based sensors, there are two Cd-Te waffers bump-bound to every
Pilatus-module, with a gap of xx pixels between the wafers.

This detector is sold with a detector-PC which is in charge of saving the
the images. 
This detector-PC comes with a 10Gbit/s network card and is directly connected to
the data analysis server.

The detector is advertized to operate at 250 frames per seconds. 
Each frame being composed of 24 pilatus modules, this makes 2.4 megapixel
images, which are 32-bits integers (actually the dynamic range is 20 bits).
If one saves the images as uncompressed raw stream this makes 19.2 gigabits per
second to transfer.

\subsubsection{Data analysis computer}


\section{Parallel decompression of CBF images}



\section{Outlook}


\section{Conclusion}

 
\ack{Acknowledgements}
This work was carried out on the request from the Materials diffraction beamline 
at ESRF (ID15a), I would like to thank Gavin Vaughan and Marco Di Michiel, the
scientists of the beamline for providing some test data.
In the instrumentation services and development division (ISDD) from ESRF  we
would like to thank V. Armando Solé, head of data analysis unit and leader of 
the \textit{silx} project, and all our colleagues from the \textit{silx}
project:
Thomas Vincent, Henri Payno, Damien Naudet, Pierre Knobel and Valentin Valls for
their support and ideas.

\bibliographystyle{iucr}
\bibliography{biblio}


\end{document}
